\documentclass[a4paper,11pt]{article}

\usepackage[english]{babel}
\usepackage[T1]{fontenc}
\usepackage[utf8]{inputenc}
\usepackage{amsmath}
\usepackage{graphicx}
\usepackage{xcolor}
\usepackage[colorinlistoftodos]{todonotes}
\usepackage{enumitem}
\usepackage{listings}
\lstset{basicstyle=\footnotesize\ttfamily,breaklines=true}
\lstset{frame=tb,language=C,numbers=left,showstringspaces=false}
\renewcommand{\lstlistingname}{Code Block}

\usepackage{geometry}
\geometry{total={210mm,297mm},
left=25mm, right=25mm,
bindingoffset=0mm,
top=20mm, bottom=20mm}

\usepackage[
  pdftitle={Homework Assignment 3},
  pdfauthor={William Jagels},
  colorlinks=true,linkcolor=blue,urlcolor=blue,citecolor=blue,bookmarks=true,
bookmarksopenlevel=2]{hyperref}

\usepackage{titlesec}
\titlelabel{\thetitle.\quad}
\renewcommand{\thesection}{\Roman{section}}

\def\code#1{\texttt{#1}}

\title{Homework Assignment 3}

\author{William Jagels}

\date{\today}

\begin{document}
\maketitle

\section{Q\&A}

\subsection{}
\begin{enumerate}
  \item \code{abort()} is supposed to halt the program in the event of an unrecoverable
    error in the program.
    By default, \code{SIGABRT} will halt and core dump.
  \item We want to make sure that the signal handler is run for \code{SIGABRT}.
    The first one lets the user's handler go to work, and hopefully that should call
    exit on its own.
    If that returns, then we have to force the default signal handler, and then use that
    to fully kill the program.
  \item The self-kill should always execute the default \code{SIGABRT} handler which
    is defined to exit the program.
    If line 27 is executed, something is terribly wrong.
  \item We don't want any other code being executed because it could rely on the integrity
    of the thing that just caused \code{abort()} to be called.
    Allowing other signals to be caught may be dangerous, so we shouldn't allow that to happen.
\end{enumerate}

\subsection{}
\begin{enumerate}
  \item This code spawns one child that spawns another child, and then kills iteslf, leaving
    the grandchild as an orphan.
    The orphan will continue with init as its parent as if it was started by init.
    The parent process can also do whatever it wants, maybe even communicate with the orphan.
  \item This code is useful if the programmer wants to start a daemon process.
    When the shell used to call the initial process exits, the grandchild process won't exit.
    The grandchild can run some sort of a server or service for the rest of the machine without
    the risk of being killed by a shell exiting.
\end{enumerate}

\subsection{}
\begin{enumerate}
  \item
    Code without condition variable.
    \begin{lstlisting}
struct msg *workq;
pthread_mutex_t qlock = PTHREAD_MUTEX_INITIALIZER;
pthread_mutex_t olock = PTHREAD_MUTEX_INITIALIZER;

void process_message() {
  struct msg *mp;

  for (;;) {
    pthread_mutex_lock(&qlock);
    while (workq == NULL) {
      pthread_mutex_unlock(&qlock);
      sleep(1);
      pthread_mutex_lock(&qlock);
    }
    mp = workq;
  }
}

void enqueue_msg(struct msg *mp) {
  pthread_mutex_lock(&qlock);
  mp->m_next = workq;
  workq = mp;
  pthread_mutex_unlock(&qlock);
}
\end{lstlisting}
\item The condition variable is better in this situation because it prevents needless unlocking
  and locking of the mutex.
  When there's nothing in workq, there will be a lot of useless work done on the cpu.

\item When \code{pthread\_cond\_wait()} is called, the mutex is unlocked and then the consumer
  thread blocks on the condition variable, allowing the producer thread to do work.

\item Sending the signal before the unlock is fine, the wait also waits for the
  lock to be unlocked by the other producer thread.
\end{enumerate}

\subsection{}
Using \code{pthread\_mutex\_trylock}, in an \code{if} block will make this work.
Since trylock will only succeed for one thread, there will only ever be one thread
that enters that branch, unless the mutex is unlocked.

\end{document}
