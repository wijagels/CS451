\documentclass[a4paper,11pt]{article}

\usepackage[english]{babel}
\usepackage[T1]{fontenc}
\usepackage[utf8]{inputenc}
\usepackage{amsmath}
\usepackage{graphicx}
\usepackage{xcolor}
\usepackage[colorinlistoftodos]{todonotes}
\usepackage{enumitem}
\usepackage{DejaVuSansMono}
\usepackage{listings}
\lstset{basicstyle=\footnotesize\ttfamily,breaklines=true}
\lstset{frame=tb,language=C,numbers=left,showstringspaces=false}
\renewcommand{\lstlistingname}{Code Block}

\usepackage{geometry}
\geometry{total={210mm,297mm},
left=25mm, right=25mm,
bindingoffset=0mm,
top=20mm, bottom=20mm}

\usepackage[
  pdftitle={Homework Assignment 1},
  pdfauthor={William Jagels},
  colorlinks=true,linkcolor=blue,urlcolor=blue,citecolor=blue,bookmarks=true,
bookmarksopenlevel=2]{hyperref}

\usepackage{titlesec}
\titlelabel{\thetitle.\quad}
\renewcommand{\thesection}{\Roman{section}}

\def\code#1{\texttt{#1}}

\title{Homework Assignment 1}

\author{William Jagels}

\date{\today}

\begin{document}
\maketitle

\section{Q\&A}

\subsection{}
\begin{lstlisting}[caption=badcode.c]
int main(int argc, char * argv[])
{
    char * type = NULL, * op1 = NULL, * op2 = NULL;
    int a = 0, b = 0;
    if (argc != 3 && argc != 4)
    {
        print_usage();
        exit(0);
    }
    type = argv[1];
    op1 = argv[2];
    op2 = argv[3];
    a = atoi(op1);
    b = atoi(op2);
    switch (type)
    {
        case "1":
            printf("Multiplication result is %d\n", mymul(a, b));
        case "2":
            printf("Division result is %d\n", mydiv(a, b));
        case "3":
            myabs(a);
            printf("Absolute value is %d\n", a);
        case "4":
            printf("Concatenation result is %s\n", myconcat(op1, op2));
    }
    return 0;
}
void print_usage()
{
    printf( "Usage: \"./a.out operation_type operand_1 [operand_2]\"\n"
            "\t operation_type :\n"
            "\t\t 1: multiply operand1 and operand2\n"
            "\t\t 2: divide operand1 by operand2\n"
            "\t\t 3: absolute value of operand1\n"
            "\t\t 4: concatenate operand1 and operand2\n");
}
int mymul(int a, int b)
{
    return a * b;
}
int mydiv(int a, int b)
{
    return a / b;
}
void myabs(int a)
{
    a = -a;
}
char * myconcat(char * a, char * b)
{
    char * str = malloc(strlen(a) + strlen(b));
    strcpy(str, a);
    strcat(str, b);
    return str;
}
\end{lstlisting}

\textbf{Problems with the above}
\begin{enumerate}
  \item
    All the functions are implicitly defined, which isn't all that safe.
    Forward declarations should be used.
  \item
    At line 12, \code{arv[3]} may point past the end of the array.
  \item
    At line 15, the switch statement has no default, making it hard for the user to figure
    out that they supplied an invalid type.
  \item
    The switch case also won't work because C only alows integers in a switch case.
    This can be fixed by using \code{switch(atoi(type))} and removing the quotes around
    each case number.
  \item
    The switch case also behave erratically without \code{break;} at the end of each case.
    All operations after the selected operation will all execute without it.
  \item
    \code{myabs} does not have any effect as it doesn't return anything.
    Also, absolute value of a is not always -a.
    Replacing \code{myabs} with \code{abs} from \code{<stdlib.h>} is a good alternative.
    This would require deleting line 22 and replacing \code{a} with \code{abs(a)} on line 23.
  \item
    \code{myconcat} allocates a block for the resultant string, but it is never freed
    when called from \code{main}.
  \item
    \code{myconcat} doesn't allocate enough space at line 52, it should allocate
    \code{strlen(a) + strlen(b) + 1} bytes because of the null terminator.
\end{enumerate}

\subsection{}
This code could be useful if \code{dest} points to a memory-mapped output device.

\subsection{}
\begin{enumerate}[label = (\alph*)]
  \item \code{int a, *b;}\\
    \code{a} is a normal integer while \code{b} is a pointer to an integer
  \item \code{double (fs[])(int);}\\
    \code{fs} is an array of functions that take an \code{int} and return a \code{double}.
    Sadly, this is not a valid type, because functions can't be stored in arrays.
    A \code{*} before \code{fs} will make this a valid type.
  \item \code{double *(*f)(double d);}\\
    \code{f} is a pointer to a function that takes a \code{double} and returns a \code{double *}
  \item \code{int *const p;}\\
    \code{p} is a constant pointer to an integer.
    The pointer address can't change, but the data it points to can be.
  \item \code{int (*cmps[10])(const void *, const void *);}\\
    \code{cmps} is an array of 10 pointers to functions that take two pointers of type void that point to constants and return \code{int}.
\end{enumerate}

\subsection{}
\begin{enumerate}[label = (\alph*)]
  \item\code{typedef double (*arg\_t[])(int);\\
  typedef void *(*ret\_t)();\\
  ret\_t f(arg\_t);}
  \item\code{void *(*(*foo)(double (*[])(int)))();}
\end{enumerate}

\begin{lstlisting}[caption=Function Pointers,label=function pointers]
#include <stdio.h>
#include <stdlib.h>

typedef double (*arg_t[])(int);
typedef void *(*ret_t)();

void *(*(*foo)(double (*[])(int)))();
double bar(int k) { return 1.0 * k; }
void *baz() { return malloc(10); }

ret_t fin(arg_t a) {
    printf("%f\n", a[0](6));
    return &baz;
}

int main() {
    foo = &fin;
    double (*asdf[])(int) = {&bar};
    free(foo(asdf)());
}
\end{lstlisting}

\subsection{}
\begin{enumerate}[label = (\alph*)]
  \item
    The compiler actually won't care because function declarations with empty arguments
    semantically means that the function can take any number of arguments of any type.
    This can be explicitly overridden by declaring \code{f} as \code{void f(void);}.
  \item
    This is true for the most part, with the exception of floating point numbers.
    If x is NaN, then \code{x == x} will always be false.
    Wonderfully enough, \code{x == NAN} doesn't even work if x is NaN.
    \code{isnan(x)} is the only safe way of checking if x is NaN since \code{NAN != NAN}.
  \item
    In some cases, this is true.
    The C standard only specifies a minimum size of 32 bits for \code{long}.
    In the case of a 32 bit \code{long} and 32 bit \code{int} there will be overflow because
    of the sign bit.
    On my machine, a \code{long} is 64 bits wide, and does not cause any overflow.
    A good way to guarantee safety is to use the \code{long long} specifier, which will always
    be at least 64 bits wide.
  \item
    \code{malloc} is a fairly expensive call, so calling it a bunch of times isn't the best idea.
    Also, calling \code{malloc} repeatedly will likely result in fragmented memory, meaning
    that memory locality will be broken.
    In addition, in 4-byte aligned systems, this will result in a 25\% utilization of the
    malloc'd blocks.
  \item
    \code{typedef} doesn't need to define any \underline{new} types.
    A typedef might simply create an alias to an already defined type, as in this nasty typedef:
    \code{typedef char* string}
\end{enumerate}

\end{document}
